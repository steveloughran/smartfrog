\section{The Fundamental Flaws of JAX-RPC}
\label{objections}

\subsection{The Object/XML Impedance Mismatch}
\label{objections:o-x}

JAX-RPC attempts to turn an XML document into Java classes, using the
type information attached to nodes. This is distinct from the kind of
mapping performed by DOM implementations, in that the classes are
``serialised'' from the XML tree, not merely created to represent it
(it is a semantic rather than syntactic mapping). This
serialisation/deserialisation is an essential part of JAX-RPC,
allowing method calls to be translated into SOAP requests, and
responses translated back into Java objects.

We believe that the term \emph{serialisation} downplays the nature of
the problem, likening it to the more tractable problem of creating a
non-portable persistence format for a class. Instead, we prefer to use
the term \emph{O/X mapping} to emphasise the similarities it has with
the heavily studied \emph{O/R mapping problem}\footnote{
\emph{"Object-relational mapping is the Vietnam of Computer Science"}
- Ted Neward.
%{\small \tt http://www.neward.net/ted/weblog/index.jsp?date=20041003#1096871640048}
}. 
Over a decade has been spent trying to map between records in relational
databases and language-level objects, and there is still no sign of an ideal
solution. There is significantly less experience in mapping between XML and
objects, and rather than drawing on the experiences of the many failed attempts
at O/R mapping, O/X mapping technologies appear destined to share a similar
evolution.

At first glance, O/X mapping appears simple: create a Java object for
each XML element, building a DAG\footnote{directed, acyclic graph}
when serialising to RPC/encoded SOAP or a tree with document/literal
messages. Read or write between attributes and class fields, bind to
children and the conversion is complete. If only it were so
straightforward. There is a fundamental difference between the type
systems of XML (especially that of XML Schema) and that of Java, which
makes any mapping both complex and brittle.

\subsubsection{Binding XML Elements to Java Classes}
\label{objections:o-x:xml-classes}

The language of XML Schema is much richer than the object model of
Java. In Java, inheritance can extend a type, and change some existing
semantics, but derivation by restriction is not explicitly
supported. Java, in common with many object oriented programming
environments, allows derived types to expand upon the capabilities of
their parents, but not to decrease them. That is, one can add attributes and
methods, but not remove them. XML schema lets one extend a type by 
restricting it, constraining attribute and element values.

This fundamental difference means that one cannot accurately model an
XSD type hierarchy in a Java class hierarchy. All one can do is
inaccurately model it.

Here, for example, a postcode is modelled by restricting a string:

%\todo{more entertaining example. something like `us-beer' is a 
%restriction of type `beer' where alcohol is a removed attribute}

\begin{verbatim}
<simpleType name="UKPostcode">
  <restriction base="xsd:string">
    <pattern value="[A-Z]{2}\d \d[A-Z]{2}"/>
  </restriction>
</simpleType>
\end{verbatim}

The actual result is going to be a simple class of type {\tt String}:
all restriction information will be lost in the transformation from
XML Schema to Java. This is a fundamental difference, and one which
would appear to remain intractable except in special cases.

\subsubsection{Mapping XML Names to Java Identifiers}
\label{objections:o-x:names}

Not all XML names can be turned into Java identifiers.  XML names may
begin with a letter in one of many Unicode languages, an ideograph or
an underscore (``\_'') . They can be followed by any of the same
characters, and also a hyphen ``-'' or a full stop ``.''. Some
examples are: {\tt schr\"odinger}, {\tt \_unknown.type-set}, and {\tt
String}.

Java identifiers almost comprise a proper subset of XML names
\footnote{XML names beginning in "xml" (any case) are reserved},
which means that the system needs to perform a non-trivial mapping
from the XML names to valid class and package names. Package names
typically being derived from namespace URLs if not overridden.

The translation is inordinately complex and brittle: whenever a new
version of Java is released, the logic must be updated to avoid new
reserved words (like {\tt assert} and {\tt enum}), or the generated
code will no longer compile in the enhanced language. Needless to say,
such an upgrade will break any existing code that linked to old
classes which made use of these names.

\subsubsection{Enumerations}
\label{objections:o-x:enum}

One specific example that deserves special mention is how
{\tt xsd:enumeration} declarations are mapped to Java. Prior to
Java 1.5, there was no explicit {\tt enum} clause in the language, so
workarounds were developed. The JAX-RPC solution is that of a common
pattern: to declare a class with a public static instance representing
a valid enumeration value. The WSDL to Java code generates such a
class, with the name of each static class taken from the name of each
value in the enumeration. 

This appears a straightforward example of how O/X mapping should
work. But what if the value of the one of the enumeration types is a
reserved word?  Our API (from section \ref{intro:experiences} contains
a lifecycle state machine like this:

\begin{verbatim}
<xsd:simpleType name="lifecycleStateEnum">
  <xsd:restriction base="xsd:string"> 
    <xsd:enumeration value="initialized"/> 
    <xsd:enumeration value="running"/> 
    <xsd:enumeration value="failed"/> 
    <xsd:enumeration value="terminated"/> 
    <xsd:enumeration value="null"/> 
  </xsd:restriction>
</xsd:simpleType>
\end{verbatim}

One element in this enumeration is reserved: {\tt null}. However, the
JAX-RPC specification states that an implementation must now enumerate
all states as {\tt value1}, {\tt value2}, and so on, for the entire
list.  The enumeration names in the Java source no longer contain any
informative value at all, other than a position number in the
set. This is inordinately brittle, as any change to the enumeration
could reorder the values, without the code detecting a change.

\subsubsection{Unportable types}
\label{objections:o-x:types}

Some Java types are by nature explicitly unportable. One would not
expect to be able to have a SOAP runtime serialise a database
connection instance and have it reconsituted in working order at the
far end. One might hope that a {\tt java.util.Hashtable} could be
translated into some structure that could be turned into a
platform-specific equivalent at the far end. But surely a {\tt
java.util.Calendar} object could be sent over the wire, as it
apparently maps so well to the {\tt xsd:dateTime} type in XML Schema.

We can certainly send such times. They are readable on the wire, and
are mapped into whatever a remote endpoint has to represent
time. Unfortunately, due to differences in expectations between Java
and .NET date/time classes, we can not guarantee that the same time
will be received at the far end. If both client and server are in the
UTC time zone all works well, but if either of them are in a different
location, hours appear to get added or removed. Clearly a different
expectation regarding time processing is taking place.

This is an insidious class of defect as it is not apparent on any
testing which takes place in the same time zone, or between Java
implementations. It is only apparent when remote callers, using
different platforms, attempt to use the service.

Our service API now uses UTC seconds since 1/1/1970, the classic {\tt
time\_t} format, as our time representation. This is no longer human
readable, but it works\footnote{It violates the spirit (if not the
letter) of SOAP's aim to be self-describing, however.}.

\subsubsection{XML Metadata and Namespaces}
\label{objections:o-x:namespaces}

As discussed in the previous sections, XML Schema provides a type
system that is much richer than that of Java. One aspect not mentioned
so far is the relationship between XML metadata, notably namespaces,
and Java classes.

The problem is essentially as follows: each node in an XML message can
have attached to it a namespace. There is no related construct in Java
which can model this accurately. The choice that is normally made is
to model it innaccurately by package names (mapping namespaces to Java
packages provides many of the problem discussed in section
\ref{objections:o-x:names}). 

The problems that typically arise are of two kinds:
\begin{enumerate}
\item Mapping an incoming message to a web service object requires
guessing the namespace of either the operation itself or its
parameters. This guessing can be wildly inaccurate when the web
service's Java interface was generated from WSDL using package
renaming.
\item When dynamic invocation is desired (service invocation without
the use of pre-built stub classes) it can be very difficult to
determine the correct namespaces for service invocations (the WSDL
typically leaves this unspecified, meaning that for JAX-RPC services
the WSDL is not a complete description of the service interface).
\end{enumerate}

\subsubsection{Message validation}
\label{objections:o-x:validation}

When a message is received, the serialised form is generated and
passed to the handlers for processing. No validation of the incoming
XML against the message schema is performed. In particular, any
restrictions on the number of times an item is required are not
checked.

This forces the implementation code to follow one of two paths:
\begin{enumerate}
\item It could ignore the problem. If the client code and functional
tests do not generate invalid messages (as is likely if they are also
all written in JAX-RPC) then the problem will not be noticed, only
only surfacing when a third party attempts to use the service.

\item The developers could write procedural logic to verify that
the Java classes representing a deserialised message have a
structure that matches their expectation based on the schema. This
requires an understanding of the schema, knowledge of the
serialisation mapping and potential trouble spots, the willingness to
write the tests to validate this extra logic, and most of all, time.
\end{enumerate}

We suspect that most services err on the side of ignorance, and do not
validate their incoming messages adequately. This brings into question
their ability to interoperate with general clients.

\subsubsection{Inadequate Mixing of XML and Serialised Data}
\label{objections:o-x:mixing}

JAX-RPC and JAXM are two different views of the world. What does JAX-RPC
do when it encounters a piece of random XML in a message? It creates a
JAXM {\tt Node} to describe that part of the tree.

From that point on, the tree below the node is permanently isolated
from the JAX-RPC model: the developer has sailed off the edge of the
JAX-RPC world, and fallen into the universe of XML. Any O/X mappings
which may exist for data within this piece of the message are now
inaccessible, all that is left is the low-level JAXM API. 

This behaviour makes it appear that incorporating arbitrary XML within
a SOAP message is not an approved action, yet the ability to easily
incorporate such XML is a key aspect of SOAP's flexibility and a key
to its being more extensible and less brittle than its predecessors.

\subsubsection{Fault processing}
\label{objections:soap-not-rmi:faults}

JAX-RPC tries to marshall Java faults over the network in such a way
that they can be reconstituted at the far end into the same fault.
This is a somewhat complex process to manage, as the class name of the
fault must be exchanged as the fault code. Since faults are often
immutable, the standard serialisation mechanism of named getter and
setter methods must be replaced by a more exotic one: getter methods
are used to extract the contents of a fault, a fault which must offer
a constructor that takes every attribute in a parameter of the same
name\footnote{This implicitly requires code to be built with debugging
information, so that the bytecode can be analysed to determine
parameter names.}.

We believe that attempting to seamlessly marshall faults is a mistaken
approach.  By propagating the still controversial ``declare all
possible faults'' rule of Java into remote interfaces, it exposes
platform implementation details. If a service could only raise a
normal {\tt SOAPFault} unless its developers explicitly declared and
implemented custom WSDL fault elements, service definition would be
platform-neutral.

%% I don't understand the relevance of this so I cut it
% As well as the marshalled faults, the specification includes one
% standard fault, {\tt SoapFaultException}.

Exposing implementation details in the service interface makes ensuring
interoperability much more difficult. We recall that interoperability
was a major reason for adopting SOAP initially, and that this is yet
another capability of SOAP's which JAX-RPC fails to deliver upon.

\subsection{SOAP is not just RPC}
\label{objections:soap-not-just-rmi}

SOAP's parentage includes XML-RPC \cite{winer:xmlrpc} and indirectly
COM/DCOM \cite{dbox:com}. It was clearly designed at its outset to be
a form of remote procedure call in XML, over HTTP. Over time, the
world-view that lead to that choice has changed. Though it is often
presented as a form of RPC, it is coming to be seen that it is more
powerful when viewed as a system where arbitrary XML documents are
exchanged between parties, potentially asynchronously, and potentially
via intermediaries.

In this world, the programming paradigms that seemed appropriate for
an RPC infrastructure look out of place. On a fast network, RPC
invocation is often a good choice, as other models of communication
are harder to code, and their benefits are not readily apparent. When
we begin to work over long-haul connections, however, or with large
content (eg fifteen megabyte attachments), the limitations of RPC
become clear.

The greatest of these is that RPC is implicity synchronous. Although
asynchronous behaviours can, with some difficulty, be introduced, this
is not the natural way for RPC to behave. As content becomes larger
and the network latency increases, the problems posed by synchronous
calls become much more acute.

Currently, our only option is to split network communication into a
separate thread from the rest of the program. While this works, it
provides the programmer no way to give the user effective feedback or
control over the communications. There is no way to receive progress
notifications or cancel an active call, even though the underlying
transport code invariably permits such features.

Again, following our principle that SOAP technologies should
attempt to work to the same goals as SOAP itself, we note that SOAP
was designed to work over long haul connections (and be simple). By
making it both difficult and complicated to work over a long
connection, JAX-RPC fails to meet these criteria for a SOAP
technology.

\subsection{SOAP is not RMI}
\label{objections:soap-not-rmi}

JAX-RPC suffers from a greater flaw than those classically associated
with RPC invocation: it tries to make the communications look like
Java RMI. Java's RMI system is a simple and effective mechanism for
connecting Java classes running on different machines. It is an
IDL-free communication mechanism, which relies on introspection to
create proxy classes and to marshall classes. It works because the
systems at both ends are Java, usually different pieces of a single
larger application. Even then, it works best if both ends are running
the same version of all classes.

With synchronised versions of common code, objects can be trivially
serialised and sent over a network connection. Exceptions are just
another type of object, and so too can be sent over the wire. There is
no need for an IDL, as Java interface declarations can perform much of
the same role. And as the recipient is a remote object, state is
automatic.  One can even keep code synchronized by using a special
class loader, one that fetches code from jointly-accessible URLs.

JAX-RPC tries to reuse many of the programming patterns of RMI. For
example, the runtime will attempt to serialise classes marked as
{\tt Serialisable}, ignoring those fields marked as
{\tt transient}. It will even serialise complex compound objects where
possible. The user appears to have a reference to something like an
object, though one that represents the current conversation with an
endpoint, not a direct endpoint proxy.

% When a message gets delivered to its endpoint, the normal handler for
% dispatching the message creates an instance of the relevant Java class
% and then dispatches the request to it. The message is turned into an
% invocation of an instance of a remote object, which certainly looks
% like RMI, even if the lifecycle of the class is radically different.

SOAP strove to overcome many of the failings of precursor technologies
like CORBA and DCOM. These technologies work well over local area
networks, and enable rich bidirectional communications, but are not
completely cross platform\footnote{Arguably for political rather than
technical reasons}, and ended up being used to produce distributed
object systems that were too tightly coupled. Recall that one of our
key hopes from adopting SOAP (section \ref{introduction}) was to
enable loose coupling between components of a distributed system.

While Java RMI provides convenience, the preceding paragraphs
should have made clear that one thing it does not provide in any way
is loose coupling. Interacting systems typically run from the same
codebase (indeed running them from different codebases can pose
significant problems). By trying to turn SOAP into RMI, we risk losing
the very things we turned to SOAP for in the first place.

% Tight coupling has proven to be a bad thing in a distributed
% system. If components are too tightly coupled, it implies that the
% separate components have merged to become one large system, one which
% cannot be treated except as a whole. And in a sufficiently large
% system, it is impossible to co-ordinate the whole \cite{deutsch,
% JINI}. The move towards SOAP and a service oriented architecture was
% advocated as a means of dealing with loose organisational coupling, by
% forcing a loose data coupling between applications.

\subsection{WSDL: an extra complication}
\label{objections:wsdl-gen}

The role of an interface definition language (IDL) has always been
twofold:

\begin{enumerate}
\item Firstly, an IDL allows the creation of a definition of \emph{the
interface} of the remote system, independent of any particular
implementation, programming language or environment. This is
``interface'' in the sense of \emph{the implementation independent
signature of the service}, and does not imply that an implementation
language needs an explicit notion of interfaces. The interface is
inherently implementation independent, and can be frozen or carefully
managed with respect to versioning.

\item Secondly, the act of writing an IDL description inherently
forces the author to define the system in terms of the portable
datatypes and operations available in the restricted language of the
IDL.  This can effectively guarantee portability, and is a significant
improvement over similar definitions in implementation languages,
which invariably contain constructs which are not portable. As such
constructs are excluded from the interface language, a portability
issue is the exception, rather than being commonplace.
\end{enumerate}

IDLs have many advantages for creating interoperable systems, yet the
generally accepted practise for working with JAX-RPC discards all
these notions. Instead of generating implementation classes from WSDL,
the WSDL description is usually generated from the implementation
classes using tools leveraging Java's Reflection API. We shall term
this process \emph{contract-last development}.

This has the following consequences.

\begin{itemize}

\item
    
There is no way to ensure that the published contract of a service
remains constant over time. Every redeployment of the service, every
upgrade of the SOAP stack or even the underlying Java runtime may
change the WSDL, and hence the interface.

\item

Some aspects of the service are not extracted from the raw signatures
of the classes and methods. For example, if a method chooses to
extract attachments from a message, that information can be hidden in
the contents of the message, instead of in the signature of the
call. The generated WSDL will hence omit any information about the
attachment needs of the service.

\item

There is no warning of portability issues before integration time.
When defining a service using an IDL, the author typically knows when
there are problems as the IDL will not compile. Yet with contract-last
development, everything may well seem to work until the service goes
live and a customer using a different language attempts to import the
WSDL and invoke the service.
    
\end{itemize}

The alternative to contract-last development is clearly \emph{contract-first
development}. Although this is the better approach from the perspective
of portability and interface stability, web service developers are not
pushed in this direction.

One of the underlying causes of this has to be the sheer complexity of
XML Schema and WSDL. The XSD type system bears minimal resemblance to
that of current object oriented languages, and WSDL itself is
over-verbose and under-readable. As evidence of this, consider the
broad variety of products that aim to make authoring XSD and WSDL
documents easier, and recall that such products were never necessary
in the IDL-era of distributed systems programming.

We note in passing that that REST systems \cite{fielding:rest} tend not to
make use of WSDL, even though it is theoretically possible. Instead
they resort to their XML type language of choice and quality
human-readable documentation. This would appear to be sub-optimal, yet
REST is growing in popularity, despite (or perhaps because of) the
lack of WSDL integration.

Returning to the desiderata for SOAP, following a contract-last process
sacrifices interoperability for ease of service development. Perhaps
WSDL is not the appropriate language for describing SOAP services (we
are certainly not enthused about it), yet the sole solution being
advocated is not a major undertaking to fix WSDL's core flaws, it is
to continue to encourage developers to hand over to their SOAP stacks
the challenge of deriving a stable and portable service interface from
the inherently unstable and unportable service implementation.

We are not proposing any changes to WSDL, merely mourning the fact
that its over-complexity discourages contract-first, contract-driven
development more aggressively than any previous IDL ever did. We do
observe that once the type declarations of a service have been moved
into their own document, WSDL becomes much more manageable and this is
a pattern of service definition which we strongly encourage.



