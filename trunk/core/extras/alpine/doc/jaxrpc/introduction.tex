\section{Introduction}
\label{introduction}

In any discussion of SOAP-based technologies, it is important to
remember the core reasons for adopting SOAP originally. We
characterise them as follows:

\begin{enumerate}
\item \textbf{simplicity}: It is intended to be easy to work with.

\item \textbf{interoperability}: It is more interoperable than binary
predecessors

\item \textbf{XML-centric}: Built on XML, and intended to integrate with
XML-based technologies.

\item \textbf{extensibility}: The envelope/header/body structure allows extra
data to be attached to a request, potentially without breaking existing systems.

\item \textbf{self-describing}: Messages can contain type definitions alongside
data, and provide human readable names.

\item \textbf{flexibility}: Participants can handle variable amounts of incoming
data.

\item \textbf{long-haul}: It is designed to work through firewalls, over HTTP.

\item \textbf{loosely-coupled}: Participants are not expected to share
implementation code.
\end{enumerate}

We will refer to these criteria throughout our discussion, as the
desiderata against which any SOAP technology should be judged.

\subsection{SOAP in Java}
\label{intro:java}

Communication with SOAP can be viewed as XML-based remote procedure calls, or as
a way of submitting XML documents to remote URLs (optionally eliciting responses
in the form of XML documents). These two different perspectives represent the
RPC-centric and message-centric viewpoints. In Java, the RPC-centric model has
become the primary model of SOAP APIs.

The Java APIs representing the two different underlying perspectives
are JAX-RPC\footnote{Java API for XML-based RPC}
\cite{spec:JAX-RPC-11} and JAXM\footnote{Java API for XML Messaging}
\cite{spec:JAX-M-11}. We look at each of these in turn.

\subsubsection{JAXM}
\label{intro:jaxm}

JAXM was written to support both basic SOAP, and more complex
scenarios like asynchronous ebXML message exchange over SOAP. This
flexibility introduces significant extra complexity to the design.
Over time, the ebXML focus of JAXM has declined, while the API itself
has been renamed SAAJ\footnote{SOAP with Attachments API for Java}
\cite{spec:SAAJ-12}.

In JAXM/SAAJ, the developer works with the SOAP message through Java
interfaces derived from DOM\footnote{Document Object Model}
\cite{spec:DOM}. These are bound to a class that represents the body
of the message, which provides various operations to manipulate the
pieces. These include accessors and manipulators for the envelope,
headers, body and any binary attachments.

JAXM does not provide significant transport support: the primary
method of receiving JAXM messages is to implement and deploy an HTTP
servlet.  The sole method of sending a message is to ask a
{\tt SOAPConnectionFactory} for a {\tt SOAPConnection} instance, and
then make a blocking call of the endpoint.

JAXM is an orphan specification. If ebXML had taken off, perhaps it
would be more popular, and message-centric SOAP development in Java
commonplace. Instead JAX-RPC is touted as the recommended way to work
with SOAP in Java. 

\subsubsection{JAX-RPC}
\label{intro:jax-rpc}

The current edition of the JAX-RPC specification is version1.1; this
is the version which is under discussion. We will look at JAX-RPC2.0 in 
ref{objections:implications:future}

In JAX-RPC, all the details of how a message was encoded are hidden, the
developer works with Java objects created automatically from the XML data using
a semi-standardised mapping process. Java classes can be automatically turned
into SOAP endpoints, with each public method in the class exported as an
operation with a request message and a response message. The message structure
is described in a WSDL file, which can be hand-written, or automatically
extracted from the Java classes through introspection.

Client side proxy classes can be generated from the WSDL files, proxy
classes which again provide a method for every operation the service supports. In
Java-Java communications, the result is that methods called on the
proxy class result in the passing of the method parameters to remote
methods on an instance of the implementation class, a behaviour that
superficially resembles Java RMI \cite{paper:RMI}. We will return to
this in section \ref{objections:soap-not-rmi}.

One good architectural feature of JAX-RPC is the \emph{handler chain};
an ordered sequence of classes which are configured to manage requests
and responses. Using the handler chain, it is possible to add support
for new SOAP headers to existing services, or to apply extra
diagnostics. The dispatch of operations to Java methods, EJB methods
or other destinations is actually implemented by specific handlers,
making the handler chain the foundation for the rest of the system.

JAX-RPC is widely implemented, both by open source projects (for
example Apache Axis \cite{apache:axis}), and by commercial vendors
like Sun, IBM and BEA. These SOAP toolkits do all implement the
appropriate version of JAXM/SAAJ to go alongside the RPC model, but
this feature is neither broadly promoted nor used. All the
\emph{evangelisation} of SOAP focuses on JAX-RPC, as do most of the
examples in the vendors' documentation.

The bias is such that, for Java development, JAX-RPC \emph{is} SOAP. 

\subsection{The Hard Lessons of Service Implementation}
\label{intro:experience}

The authors have recently been involved in developing independent
implementations of a SOAP API for deployment \cite{draft:CDDLM}. This
API, specified in a set of XML Schema (XSD) \cite{spec:XSD} and Web
Services Description Language (WSDL) files \cite{spec:WSDL-11} defines
a service endpoint which providing seven operations. These operations
permit suitably authenticated callers to deploy distributed
applications to a grid fabric.

This service was defined in a ``pure way'', by writing the XSD and
WSDL files first. This approach is believed to aid in creating a
platform-independent system, and represents current best
practise. However, the XSD file for the service messages is
approximately 2000 lines, including all the comments needed to make it
comprehensible. That it takes so many lines to describe a
relatively simple service is clearly one reason why this approach,
despite its superiority of output, is so unpopular.

Many problems were encountered turning this WSDL specification into
functional clients and servers, problems that we attribute to JAX-RPC.
In section \ref{objections} we discuss a number of the problems we
believe this work highlighted. Section \ref{objections:wsdl-gen}
outlines the particular problems we believe typical JAX-RPC oriented
approaches to WSDL generation create.

%% The subsequent sections do not establish their own relevance to 
%% the question of why JAX-RPC sucks. We either need to find a link
%% between these issues and JAX-RPC sucking, or cut them.

% One of the features of this service was that it was intended to
% support multiple deployment languages, and multiple deployment
% engines, each potentially with different capabilities and options. To
% this end, some features of the design stand out.

% Firstly, the deployment payload is arbitrary XML. It is not
% appropriate or possible for the service definition to declare what
% deployment languages are supported.

% Secondly, we defined an option mechanism, in which callers can set
% arbitrary options on a deployment. Each option can have a string,
% integer or XML value, and is identified in a set of options by a
% URI. We plan to define some normative options within the working
% group, and allow implementations to define new ones.  Following the
% example of the WebDAV and SOAP header designs, options can be ignored
% unless their \verb|mustUnderstand| attribute is set. With a few well
% known option types defined for the nested XML option, we wanted
% to treat options just like other received data, and use the XML to Java
% mapping facility of the classes. 

% Thirdly, we need to raise lifecycle event notifications with
% callers. In the absence of a single, stable notification protocol, we
% chose to allow services to support multiple mechanism (again, each
% defined by arbitrary inline XML).



