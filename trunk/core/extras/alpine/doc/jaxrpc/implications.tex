\section{Implications}
\label{implications}

We believe that only two categories of web service developer exist:
those who are comfortable with XML and want to work with it, and those
who aren't but end up doing so anyway. JAX-RPC provides a sugar coated
wrapping that encourages developers who are relatively unfamiliar with
XML to bite. Yet, as anyone who has written a web service of any
complexity knows, the XML must be faced and understood eventually. In
practise, the task of creating a real web service is made more
difficult, not less, by the huge volume of code JAX-RPC introduces
into a project.

JAX-RPC only superficially benefits developers who do not want to work
with XML: by hiding all the details, and giving developers a model of
remote method calls via serialised Java graphs, JAX-RPC makes it
harder to write true, interoperable SOAP services. Not only that, but
it introduces the O/X mapping problem, while retaining an invocation
model that is inappropriate for long-distance networks and slow
communications.

We argue that JAX-RPC greatly complicates users' software by
introducing a complex and fickle serialisation system. The generation
of WSDL from Java code, which JAX-RPC encourages, makes it very
difficult to maintain version consistency of an interface, and
creates significant interoperability problems.

On top of all of this, for users who do want to work with the XML
(typically those whose first project did not!) JAX-RPC is
inappropriate because it hides everything. Trying to integrate custom
XML documents with JAX-RPC serialisations is possible, but very hard
work. In Apache Axis, DOM trees get recreated when assigning or
extracting them from {\tt SoapMessageElement} implementations.


\subsection{The Future}
\label{implications:future}

JAX-RPC has become a cornerstone of Enterprise Java
\cite{spec:J2EE-14}, alongside RMI and RMI-over-CORBA. That is not by
itself a bad thing, but we believe that it creates the misconception
that developers can trivially migrate from RMI to web services. If
they attempt to do, they will fall into the traps that JAX-RPC creates
for them.

The forthcoming 2.0 release of the JAX-RPC specification promises to
correct some of these flaws, it is unclear whether it corrects
sufficiently many of them. An alternate O/X mapping, JAX-B, the Java
Architecture for XML Binding, is introduced which is a compile-time
declaration of what XML is to be expected, but is independent of the
SOAP stack. The 2.0 release also retains the core metaphor of service
calls as method invocation, with the payload of most invocations
being Java objects that are somehow mapped to XML content. The
automated generation of WSDL from Java source is retained, despite
this approach having been shown to be fundamentally flawed.

We understand the rationale for much of this. Working with raw XML is
hard.  Writing good XML Schema documents is hard. WSDL is exceedingly
painful to work with. However, we believe that if developers do not
create the XSD and WSDL definitions of their service, they will never
have control of the messages that get sent over the wire, and that
without that control, interoperability and loose coupling will remain
out of reach.

